\chapter{Ontwerp met Apidog}
\label{ch:apidog}

In dit hoofdstuk wordt het gebruik van \textbf{Apidog} toegelicht voor het ontwerpen van de API. Apidog maakt het mogelijk om OpenAPI-specificaties visueel op te stellen, te valideren en automatisch te documenteren, wat het ontwikkelproces gestructureerd en transparant maakt.

\section{Gebruik van Apidog}

Met Apidog werden alle API-endpoints, datamodellen en parameters opgesteld volgens de OpenAPI-standaard. Hieronder is een voorbeeld van een datamodel in dogApi



\section{OpenAPI-specificatie}
Hieronder staat de volledige OpenAPI-specificatie die gegenereerd is met Apidog. Deze beschrijft alle endpoints, request- en responseformaten, en validatieregels voor de API.

\inputminted[
fontsize=\scriptsize,
breaklines,
linenos,
tabsize=2
]{yaml}{openapi.yaml}