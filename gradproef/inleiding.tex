%%=============================================================================
%% Inleiding
%%=============================================================================

\chapter{\IfLanguageName{dutch}{Inleiding}{Introduction}}%
\label{ch:inleiding}

Deze graduaatsproef richt zich op het ontwikkelen van een RESTful API met Spring Boot voor het aanmaken en beheren van Dungeons \& Dragons-personages. Dungeons \& Dragons (D\&D) is een populair rollenspel waarbij spelers unieke personages creëren met eigen rassen, klassen en eigenschappen. De API ondersteunt dit proces digitaal en gestructureerd.

Het project maakt gebruik van Spring Boot, een modern Java-framework dat binnen deze toepassing voor het eerst wordt gebruikt. De API is ontworpen volgens het documentation-first-principe met behulp van Apidog, wat zorgt voor duidelijke en consistente documentatie van de verschillende functionaliteiten.

De toepassing biedt ondersteuning voor het aanmaken, bewerken en verwijderen van personages, inclusief rassen, klassen en gebruikersauthenticatie. Geavanceerde spelmechanismen zoals gevechten of campagnes vallen buiten de scope.

De centrale onderzoeksvraag luidt: Hoe kan met behulp van Spring Boot een goed gestructureerde en bruikbare API ontwikkeld worden voor het beheren van D\&D-personages?

Het doel is een functionele en uitbreidbare backendtoepassing te realiseren die voldoet aan moderne ontwikkelingsprincipes.


\end{itemize}

\section{\IfLanguageName{dutch}{Probleemstelling}{Problem Statement}}%
\label{sec:probleemstelling}

Het beheren van personages in Dungeons \& Dragons gebeurt vaak handmatig of via tools die beperkt, betalend of moeilijk uitbreidbaar zijn. Veel spelers en spelbegeleiders (Dungeon Masters) gebruiken losse documenten, spreadsheets of niet-gecentraliseerde websites, wat leidt tot fouten, onduidelijkheden en verlies van gegevens.

Voor kleine spelgroepen, individuele D\&D-spelers en ontwikkelaars van digitale D&D-tools is er behoefte aan een eenvoudige, uitbreidbare en gratis backendoplossing waarmee personages digitaal kunnen worden aangemaakt, bewerkt en bewaard. Vooral gebruikers die een eigen tool, app of interface willen bouwen, hebben nood aan een goed gestructureerde, open API als basis.

Door een gebruiksvriendelijke en goed gedocumenteerde REST API te voorzien, kan deze toepassing een duidelijke meerwaarde bieden aan deze specifieke doelgroep, die vandaag vaak afhankelijk is van gefragmenteerde of commerciële oplossingen.

\section{\IfLanguageName{dutch}{Onderzoeksvraag}{Research question}}%
\label{sec:onderzoeksvraag}

Hoe kan met behulp van Spring Boot een uitbreidbare en goed gedocumenteerde API ontwikkeld worden die het aanmaken en beheren van Dungeons \& Dragons-personages ondersteunt voor kleine spelgroepen en individuele spelers?

\section{\IfLanguageName{dutch}{Onderzoeksdoelstelling}{Research objective}}%
\label{sec:onderzoeksdoelstelling}

Het doel van deze graduaatsproef is het ontwikkelen van een functionele en uitbreidbare RESTful API met behulp van Spring Boot, waarmee Dungeons & Dragons-personages kunnen worden aangemaakt, beheerd en verwijderd. De API moet logisch en intuïtief opgebouwd zijn, zodat gebruikers zoals spelers en ontwikkelaars er gemakkelijk mee kunnen werken. Daarnaast is het belangrijk dat de structuur modulair en flexibel is, zodat in de toekomst extra spelregels en functionaliteiten eenvoudig kunnen worden toegevoegd. Om de persoonlijke gegevens van gebruikers te beschermen, wordt er een werkende gebruikersauthenticatie geïmplementeerd. De API zal volledig en duidelijk gedocumenteerd worden via Apidog volgens het documentation-first principe, waardoor externe ontwikkelaars de API eenvoudig kunnen gebruiken en integreren. Dit project wordt gerealiseerd als een proof of concept dat de basisfunctionaliteiten operationeel toont en kan dienen als uitgangspunt voor verdere ontwikkeling of integratie met frontend-applicaties. Met deze graduaatsproef wordt aangetoond dat het mogelijk is om met Spring Boot een degelijke backend voor Dungeons & Dragons-personagebeheer te realiseren die aansluit bij de behoeften van kleine spelgroepen en individuele spelers.

\section{\IfLanguageName{dutch}{Opzet van deze graduaatsproef}{Structure of this associate thesis}}%
\label{sec:opzet-graduaatsproef}

% Het is gebruikelijk aan het einde van de inleiding een overzicht te
% geven van de opbouw van de rest van de tekst. Deze sectie bevat al een aanzet
% die je kan aanvullen/aanpassen in functie van je eigen tekst.

De rest van deze graduaatsproef is als volgt opgebouwd:

In Hoofdstuk~\ref{ch:stand-van-zaken} wordt een overzicht gegeven van de stand van zaken binnen het onderzoeksdomein, op basis van een literatuurstudie.

In Hoofdstuk~\ref{ch:methodologie} wordt de methodologie toegelicht en worden de gebruikte onderzoekstechnieken besproken om een antwoord te kunnen formuleren op de onderzoeksvragen.

% TODO: Vul hier aan voor je eigen hoofstukken, één of twee zinnen per hoofdstuk

In Hoofdstuk~\ref{ch:conclusie}, tenslotte, wordt de conclusie gegeven en een antwoord geformuleerd op de onderzoeksvragen. Daarbij wordt ook een aanzet gegeven voor toekomstig onderzoek binnen dit domein.