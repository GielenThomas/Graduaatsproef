\chapter{\IfLanguageName{dutch}{Stand van zaken}{State of the art}}%
\label{ch:stand-van-zaken}

% Tip: Begin elk hoofdstuk met een paragraaf inleiding die beschrijft hoe
% dit hoofdstuk past binnen het geheel van de graduaatsproef. Geef in het
% bijzonder aan wat de link is met het vorige en volgende hoofdstuk.

% Pas na deze inleidende paragraaf komt de eerste sectiehoofding.

\section{Stand van Zaken}

Dit hoofdstuk geeft een overzicht van de huidige stand van zaken rond digitale hulpmiddelen voor D\&D Beyond \autocite{Bradford2025}. De inhoud bouwt voort op de inleiding en spitst zich toe op de technologische ontwikkelingen die het spelen van D\&D Beyond digitaal ondersteunen. Zo wordt de lezer volledig geïnformeerd over de state-of-the-art op dit gebied, zodat die het verdere verhaal zonder aanvullende voorkennis kan volgen.

Digitale platforms zoals \textbf{D\&D Beyond} bieden uitgebreide functionaliteiten voor het creëren en beheren van personages, het bijhouden van campagnes en het integreren van digitale dobbelstenen en kaarten \autocite{Bradford2025}. Andere tools zoals \textbf{Roll20} \autocite{Melzer2024} en \textbf{Fantasy Grounds} stellen spelers en Dungeon Masters in staat virtuele tafelopstellingen te gebruiken met geavanceerde functies, waaronder dynamische kaarten, automatische dobbelsteenrollen en geïntegreerde chatfunctionaliteit \autocite{Hall2015}. Hoewel deze platforms een robuuste gebruikerservaring bieden, zijn ze vaak gesloten systemen met beperkte mogelijkheden tot aanpassing en integratie, wat de vraag naar open en uitbreidbare oplossingen versterkt.

Voor de ontwikkeling van webservices wordt vaak gekozen voor \textbf{RESTful API's}, die stateless communicatie tussen client en server mogelijk maken en zo schaalbaarheid en eenvoud in het gebruik bevorderen \autocite{restfull}. In de Java-omgeving is \textbf{Spring Boot} een gangbaar framework voor het snel opzetten van dergelijke API's, doordat het tal van ingebouwde configuraties en ontwikkeltools aanbiedt \autocite{Pratik2024}. Hierdoor kunnen ontwikkelaars vlot een robuuste backend creëren die gemakkelijk integreert met verschillende frontend-applicaties en systemen.

Een moderne methodologie in API-ontwikkeling is de \textit{API-Design First} benadering, waarbij het ontwerp van de API voorafgaat aan de implementatie. Deze werkwijze bevordert duidelijkheid en verbetert de samenwerking tussen ontwikkelaars en andere betrokkenen \autocite{apidog}. Het platform \textbf{Apidog} ondersteunt deze aanpak door hulpmiddelen te bieden voor visueel ontwerpen, testen en documenteren van API's, wat bijdraagt aan snellere ontwikkeling en hogere kwaliteit.

Tot slot is beveiliging een essentieel aandachtspunt bij API-ontwikkeling, vooral bij het verwerken van gevoelige data. Een veelgebruikte authenticatiemethode is het gebruik van \textbf{JSON Web Tokens (JWT)}, die gebruikers op een veilige en efficiënte wijze authenticeren en autoriseren zonder dat servers sessie-informatie hoeven op te slaan \autocite{Gordadze}.




