%%=============================================================================
%% Conclusie
%%=============================================================================

\chapter{Conclusie}%
\label{ch:conclusie}

% TODO: Trek een duidelijke conclusie, in de vorm van een antwoord op de
% onderzoeksvra(a)g(en). Wat was jouw bijdrage aan het onderzoeksdomein en
% hoe biedt dit meerwaarde aan het vakgebied/doelgroep? 
% Reflecteer kritisch over het resultaat. In Engelse teksten wordt deze sectie
% ``Discussion'' genoemd. Had je deze uitkomst verwacht? Zijn er zaken die nog
% niet duidelijk zijn?
% Heeft het onderzoek geleid tot nieuwe vragen die uitnodigen tot verder 
%onderzoek?

Deze graduaatsproef richtte zich op het ontwikkelen van een RESTful API voor het aanmaken en beheren van Dungeons \& Dragons-personages met behulp van het Spring Boot framework en een OpenAPI-gedreven aanpak. Het doel was om een efficiënte, schaalbare en onderhoudbare API te realiseren met moderne technologieën.

Door gebruik te maken van Apidog voor het opstellen van de OpenAPI-specificatie en het automatisch genereren van request- en responseklassen, kon een consistente en gestructureerde API worden gebouwd. Dit verminderde implementatiefouten en zorgde voor een goede afstemming tussen documentatie en code. De implementatie van controllers, services en repositories volgde een duidelijke scheiding van verantwoordelijkheden, wat de onderhoudbaarheid en uitbreidbaarheid bevordert.

MapStruct werd ingezet om DTO’s om te zetten naar domeinmodellen, wat het ontwikkelproces vereenvoudigde en de leesbaarheid verbeterde. Voor data-opslag werd JPA gebruikt en Liquibase zorgde voor gecontroleerd databasebeheer via versiebeheer en migraties, wat de betrouwbaarheid verhoogde.

Het resultaat is een werkende API die voldoet aan de functionele eisen, inclusief gebruikersauthenticatie en ondersteuning voor diverse personageklassen en rassen. De modulaire opbouw en het gebruik van gangbare technologieën maken toekomstige uitbreidingen eenvoudiger.

Er zijn echter ook beperkingen vastgesteld. Zo ontbreken uitgebreide performancetests, wat nodig is om de schaalbaarheid te garanderen.

De resultaten kwamen grotendeels overeen met de verwachtingen, maar er ontstonden ook nieuwe vragen, zoals hoe de API beter kan omgaan met hoge belasting en hoe integratie met andere Dungeons \& Dragons-tools kan plaatsvinden.

Kortom, deze graduaatsproef leverde een concrete en toepasbare oplossing die bijdraagt aan het vakgebied door een praktische toepassing van moderne Java-technologieën en een OpenAPI-gedreven ontwikkelproces. Voor toekomstig werk is er ruimte voor een front-end voor deze back-end te ontwikkelen en veredere uitbrijdingen voor de back-en te schrijven zoals bevoorbeeld het moegelijk te maken om de characters te laten levelen.

