%%=============================================================================
%% Samenvatting
%%=============================================================================

% TODO: De "abstract" of samenvatting is een kernachtige (~ 1 blz. voor een
% thesis) synthese van het document.
%
% Een goede abstract biedt een kernachtig antwoord op volgende vragen:
%
% 1. Waarover gaat de graduaatsproef?
% 2. Waarom heb je er over geschreven?
% 3. Hoe heb je het onderzoek uitgevoerd?
% 4. Wat waren de resultaten? Wat blijkt uit je onderzoek?
% 5. Wat betekenen je resultaten? Wat is de relevantie voor het werkveld?
%
% Daarom bestaat een abstract uit volgende componenten:
%
% - inleiding + kaderen thema
% - probleemstelling
% - (centrale) onderzoeksvraag
% - onderzoeksdoelstelling
% - methodologie
% - resultaten (beperk tot de belangrijkste, relevant voor de onderzoeksvraag)
% - conclusies, aanbevelingen, beperkingen
%
% LET OP! Een samenvatting is GEEN voorwoord!

%%---------- Nederlandse samenvatting -----------------------------------------
%
% TODO: Als je je graduaatsproef in het Engels schrijft, moet je eerst een
% Nederlandse samenvatting invoegen. Haal daarvoor onderstaande code uit
% commentaar.
% Wie zijn/haar graduaatsproef in het Nederlands schrijft, kan dit negeren, de inhoud
% wordt niet in het document ingevoegd.

\IfLanguageName{english}{%
\selectlanguage{dutch}
\chapter*{Samenvatting}
\lipsum[1-4]
\selectlanguage{english}
}{}

%%---------- Samenvatting -----------------------------------------------------
% De samenvatting in de hoofdtaal van het document

\chapter*{\IfLanguageName{dutch}{Samenvatting}{Abstract}}

Deze graduaatsproef behandelt de ontwikkeling van een backend API voor het creëren en beheren van Dungeons \& Dragons-personages, met behulp van het Java-framework Spring Boot. Het onderwerp is gekozen vanwege de groeiende populariteit van digitale hulpmiddelen in rollenspellen en de behoefte aan flexibele, uitbreidbare oplossingen voor spelers en ontwikkelaars.

Het onderzoek richt zich op de vraag hoe een API ontworpen en geïmplementeerd kan worden die verschillende karakterklassen, rassen en statistieken ondersteunt, terwijl ook gebruikersauthenticatie is geïntegreerd. De doelstelling was het ontwikkelen van een functioneel prototype dat de kernfunctionaliteiten voor het beheren van personages omvat.

De methodologie bestond uit een documentation-first aanpak, waarbij met behulp van Apidog een OpenAPI-specificatie werd opgesteld voorafgaand aan de implementatie. Dit zorgde voor duidelijke richtlijnen tijdens de ontwikkeling en maakte de API goed testbaar en uitbreidbaar.

De resultaten tonen aan dat met Spring Boot een efficiënte en goed gestructureerde API kan worden gerealiseerd die voldoet aan de eisen van de onderzoeksvraag. Het prototype biedt een stabiele basis voor verdere uitbreiding, zoals het toevoegen van een gebruikersinterface of extra spelmechanismen.
